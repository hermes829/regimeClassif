%!TEX TS-program = xelatex
% NOTE: as of 17 Sept 2012, this compiles in xelatex

\documentclass{beamer}
% \usetheme{Madrid} % My favorite!
% \usetheme{Antibes}
% \usetheme{Bergen} % This template has nagivation on the left
% \usetheme{Berkeley} % Nice details
%\usetheme{Berlin}
% \usetheme{Boadilla} % Pretty neat, soft color.
% \usetheme{Copenhagen} % sim to default, pretty sure Sunshine uses this
% \usetheme{Darmstadt} % not so good
% \usetheme{Dresden} % sim to Berlin
% \usetheme{default} 
% \usetheme{Frankfurt} % Similar to the default 
% \usetheme{Goettingen} % Navigation on right
% \usetheme{Hannover} % Navigation on left, soft color
% \usetheme{Ilmenau}
% \usetheme{Juanlespins} % don't have this .sty
% \usetheme{Madrid}
% \usetheme{Malmoe} % pretty good. No stuff on top, sim to Warsaw on bottom
% \usetheme{Marburg} % Nice Navigation on right
% \usetheme{Montpellier} % yuck
% \usetheme{Paloalto} % don't have this .sty
% \usetheme{Pittsburg} % don't have this .sty
% \usetheme{Rochester} % very plain
%\usetheme{Singapore} % similar to default
\usetheme{Warsaw}
%with an extra region at the top.
%\usecolortheme{seahorse} % Simple and clean template
% Uncomment the following line if you want %
%page numbers and using Warsaw theme%
%\setbeamertemplate{footline}[page number]
%\setbeamercovered{transparent}
\setbeamercovered{invisible}
% To remove the navigation symbols from 
% the bottom of slides%
\setbeamertemplate{navigation symbols}{} 
%\setbeamercovered{transparent}
%\usecolortheme{albatross}
%\usecolortheme{beetle}
%\usecolortheme{crane}
%\usecolortheme{dove}
%\usecolortheme{fly}
%\usecolortheme{seagull}
%\usecolortheme{wolverine}
%\usecolortheme{beaver} % I like this one
%
\usepackage{coordsys} % for number lines
\usepackage{graphicx}
\usepackage{multirow}
\usepackage{caption}
\usepackage{subfig}
\usepackage{tikz}
%\usepackage{bm}         % For typesetting bold math (not \mathbold)
%\logo{\includegraphics[height=0.6cm]{yourlogo.eps}}
%

% font customization
% \usepackage{mathspec}
% \usepackage{xunicode}
% \usepackage{xltxtra}
% \setmainfont{Gill Sans}
% \setmathsfont(Digits,Latin,Greek){Gill Sans}

%%%%%%%%%%%%%%%%%%%%%%%%%%%%%%%%%%%%%%%%%
\title[Regime Classification \hspace{14em} \insertframenumber/
\inserttotalframenumber]{Classifying Regimes through SVM}
\author{Shahryar Minhas}
\institute[Duke University]
{
{\emph{sfm12@duke.edu}} \\
\medskip
Duke University 
}
\date{\today}

% \graphicspath{{/Users/janus829/Dropbox/Research/wardprojects/regimeclassif/}}

\begin{document}

%%%%%%%%%%%%%%%%%%%%%%%%%%%%%%%%%%%%%%%%%
\begin{frame}
\titlepage
\end{frame}
%%%%%%%%%%%%%%%%%%%%%%%%%%%%%%%%%%%%%%%%%

%%%%%%%%%%%%%%%%%%%%%%%%%%%%%%%%%%%%%%%%%
\begin{frame}
\frametitle{Goal}

Develop probabilistic measures of:

\begin{itemize}
	\item Democracy
	\item Monarchical
	\item Military
	\item Party
\end{itemize}

\end{frame}
%%%%%%%%%%%%%%%%%%%%%%%%%%%%%%%%%%%%%%%%%

%%%%%%%%%%%%%%%%%%%%%%%%%%%%%%%%%%%%%%%%%
\begin{frame}
\frametitle{Textual Data}
\scriptsize{
	\begin{itemize}
		\item 
		\item 
	\end{itemize}
	}
\end{frame}
%%%%%%%%%%%%%%%%%%%%%%%%%%%%%%%%%%%%%%%%%

%%%%%%%%%%%%%%%%%%%%%%%%%%%%%%%%%%%%%%%%%
\begin{frame}
\frametitle{Supervised Approach}

\begin{itemize}
	\item Naive Bayes
	\item Support Vector Machines (SVM)
\end{itemize}

\end{frame}
%%%%%%%%%%%%%%%%%%%%%%%%%%%%%%%%%%%%%%%%%

%%%%%%%%%%%%%%%%%%%%%%%%%%%%%%%%%%%%%%%%%
\begin{frame}
\frametitle{Labeled Data}

\begin{itemize}
	\item Democracy = 1 if
	\begin{itemize}
		\item Polity = 10
		\item FH = Free
	\end{itemize}
	\item Monarchical
	\begin{itemize}
		\item
		\item
	\end{itemize}
	\item Military
		\begin{itemize}
		\item
		\item
	\end{itemize}
	\item Party
		\begin{itemize}
		\item
		\item
	\end{itemize}
\end{itemize}

\end{frame}
%%%%%%%%%%%%%%%%%%%%%%%%%%%%%%%%%%%%%%%%%

%%%%%%%%%%%%%%%%%%%%%%%%%%%%%%%%%%%%%%%%%
\begin{itemize}
\frametitle{Results}

\begin{table}[ht]
\centering
\begin{tabular}{lcccc}
\hline\hline
~ & Precision & Recall & F-score & Accuracy \\
Naive Bayes &  &  &  &  & \\
SVM &  &  &  &  & \\
\hline\hline
\end{tabular}
\end{table}

\end{itemize}
%%%%%%%%%%%%%%%%%%%%%%%%%%%%%%%%%%%%%%%%%

% End of slides
\end{document} 